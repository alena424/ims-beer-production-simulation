% 1. projekt do Počítačové komunikace a sítě
% Jan Šorm (xsormj00), 2018
\documentclass[11pt,a4paper]{article}
\usepackage[utf8]{inputenc}
\usepackage[czech]{babel}
\usepackage[T1]{fontenc}
\usepackage{amsmath}
\usepackage{amsfonts}
\usepackage{amssymb}
\usepackage{fancyvrb}
\usepackage{float}
\usepackage{graphicx}
\usepackage{picture}
\usepackage[left=2cm,right=2cm,top=2.5cm,bottom=2cm]{geometry}
\author{Jan Šorm}
\begin{document}
\pagestyle{headings}
%%%%%%%%%%%%%%------------Titulni strana------------%%%%%%%%%%%%%%
\begin{titlepage}
	\begin{center}
		{\Huge\textsc{Vysoké učení technické v~Brně}}\\
		\medskip
		{\huge\textsc{Fakulta informačních technologií}}\\
		\vspace{\stretch{0.382}}
		{\LARGE Modelování a simulace}\\
		\medskip
		{\Huge Zadání č. 6 - Výroba piva}\\
		\vspace{\stretch{0.618}}
	\end{center}
	{\Large\today \hfill Šorm Jan, Tesařová Alena}
\end{titlepage}

\tableofcontents
\newpage

\section{Úvod}
Tato technická zpráva vznikla jako součást projektu do předmětu Modelování a simulace na škole Vysoké učení technické v Brně. V této zprávě je popsán simulační model \cite[str. 7]{pred} výroby piva ve společnosti Brněnská pivovarnická společnost s.r.o. \cite{pivo}. Důležitou součástí této zprávy jsou pak simulační experimenty, které slouží ke zjištění, zda nelze dosáhnout zefektivnění výroby.

\subsection{Autoři a zdroje informací}
Autory projektu jsou Jan Šorm a Alena Tesařová. Informace jsou čerpány z exkurze v pivovaru, které se zúčastnili oba autoři, osobního rozhovoru s panem Petrem Hauskrechtem po exkurzi a pozdější e-mailové korespondence se společností.

\subsection{Ověření validity modelu}
Většina našich informací pochází přímo od pana Hauskrechta, který je spoluvlastníkem společnosti a zároveň i dlouholetý sládek, takže by celý výrobní proces se všemi parametry měl být velmi přesný. Samotné výsledky našeho experimentu ohledně roční výroby, pak souhlasili s údaji, které nám sdělil pan Hauskrecht.

	
\section{Rozbor tématu a použitých metod}
Tématem práce je simulace \cite[str. 8]{pred} výroby piva ve společnosti Brněnská pivovarnická společnost s.r.o. Pro výrobu piva jsou potřeba 4 základní ingredience: slad, voda, chmel a pivovarské kvasnice, které má pivovar vždy k dispozici. Výroba piva pak probíhá po várkách, kde z jedné várky se vyrobí 20 hl piva. Během jedné dvanáctihodinové směny pak proběhne pro jednu várku vystírání, rmutování, scezování, chmelovar a zchlazování a várka se přesune do kvasného tanku, kde začne probíhat několikadenní kvašení, než je tank připraven ke stáčení.

Kvasných tanků je celkem 23 o kapacitě 48 hl, takže do každého tanku se vlezou dvě várky a zbývající prostor je využit na pěnu. Jelikož je v tanku místo na dvě várky, tak se vždy vaří po sobě dvě várky stejného druhu piva. 

V pivovaru se vyrábí celkem 4 různé druhy piva. Každý druh se kvasí jinak dlouho a také tvoří jinačí procentuální zastoupení v celkové výrobě:
\begin{itemize}
  \item 10$^\circ$ -- 30 dní kvašení -- 13 \% celkové výroby
  \item 11$^\circ$ -- 35 dní kvašení -- 70 \% celkové výroby
  \item 12$^\circ$ -- 45 dní kvašení -- 10 \% celkové výroby
  \item 16$^\circ$ -- 90 dní kvašení -- 7 \% celkové výroby
\end{itemize}

Po skončení kvašení je celý tank připraven ke stáčení. Jeden tank se stáčí 6 hodin a další 4 hodiny se pak čistí než je použitelný pro další naplnění. Nová várka se začne dělat jen v případě, že existuje alespoň jeden čistý kvasný tank nebo tank, do kterého lze doplnit ještě jednu várku.

Směny na vaření jsou každý všední den dvě, v sobotu probíhá pouze denní a v neděli probíhá povinné čištění tanků na vaření. Směny na stáčení pak jsou jednou každý všední den.



\subsection{Použité postupy}


\subsection{Původ použitých postupů}


\section{Koncepce modelu}


\section{Architektura simulačního modelu}


\section{Podstata simulačních experimentů a jejich průběh}

   
\section{Shrnutí simulačních experimentů a závěr}

   
\newpage
\begin{thebibliography}{Per00}
	\bibitem[1]{pred} Peringer P., Hrubý M.:
    \emph{Modelování a simulace [online]. [vid. 5. prosince 2018]. Dostupné z: https://www.fit.vutbr.cz/study/courses/IMS/public/prednasky/IMS.pdf }
	\bibitem[2]{simlib} Peringer P.:
    \emph{SIMLIB/C++ [online]. [vid. 5. prosince 2018]. Dostupné z: http://www.fit.vutbr.cz/\~{}peringer/SIMLIB/ }
	\bibitem[3]{pivo} \emph{Brněnská pivovarnická společnost s.r.o. [online]. [vid. 5. prosince 2018]. Dostupné z: https://www.hauskrecht.cz/ }
\end{thebibliography}
	
\end{document}
